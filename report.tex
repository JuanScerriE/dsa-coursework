\documentclass[12pt]{article}

% PACKAGES

% \usepackage[siunitx]{circuitikz}
% \usepackage{pgfornament}
% \usepackage{fancyvrb}
\usepackage[
top	= 2.50cm,
bottom = 2.50cm,
left = 2cm,
right = 2cm,
marginparsep = 0pt,
marginparwidth =0pt]{geometry}
\usepackage{fancyhdr}
\usepackage{float}
\usepackage{amsmath}
\usepackage{amssymb}
\usepackage{textcomp}
\usepackage{ulem}
\usepackage{contour}
\usepackage{graphicx}
\usepackage{svg}
\usepackage{xcolor}
\usepackage[T1]{fontenc}
\usepackage[utf8]{inputenx}
\usepackage[unicode]{hyperref}
\usepackage[shortlabels]{enumitem}
\usepackage{booktabs}
\usepackage{bookmark}
\usepackage{listings}
\usepackage{xcolor}
\usepackage{tocloft}

% MACROS & DEFS

% \newcommand{\splitmark}[2]{(#1)
% 	node[circle, fill, inner sep=0pt, outer sep=0pt,
% 	minimum size=2*#2]{}%
% }

% \newcommand{\jumpD}[2]{([shift={(0mm,#2)}]#1)
% 	arc[start angle=90, end angle=-90, radius=#2]%
% }

% \newcommand{\jumpU}[2]{([shift={(0mm,-#2)}]#1)
% 	arc[start angle=-90, end angle=90, radius=#2]%
% }

% \newcommand{\jumpR}[2]{([shift={(-#2,0)}]#1)
% 	arc[start angle=180, end angle=0, radius=#2]%
% }

% \newcommand{\jumpL}[2]{([shift={(#2,0)}]#1)
% 	arc[start angle= 0, end angle= 180, radius= #2]%
% }

\DeclareRobustCommand{\ul}[1]{%
	\uline{\phantom{#1}}%
	\llap{\contour{white}{#1}}%
}

\renewcommand{\ULdepth}{1.8pt}
\contourlength{0.8pt}

\setlength{\parindent}{0em}
\setlength{\parskip}{0.75em}

\definecolor{codegreen}{RGB}{0,135,0}
\definecolor{codegray}{RGB}{135,135,135}
\definecolor{codemagenta}{RGB}{215,0,135}
\definecolor{codepurple}{RGB}{135,0,175}
\definecolor{backcolour}{RGB}{238,238,238}

% PACKAGE CONFIG

% \usetikzlibrary{calc, arrows, positioning, circuits.logic.US}
\graphicspath{ {./images/} }

\lstdefinestyle{code}{
	basicstyle=\ttfamily\small,
	commentstyle=\color{codegray}\itshape,
	keywordstyle=\color{codepurple},
	stringstyle=\color{codegreen},
	aboveskip=10pt,
	captionpos=b,
	abovecaptionskip=12.5pt,
	breaklines=true,
	numbers=none,
	frame=tb,
	framesep=5pt,
	keepspaces=true,
	showspaces=false,
	showstringspaces=false,
	breakatwhitespace=false,
	tabsize=2
	showtabs=false,
}

\lstset{style=code}

% Set dots for table of contents
\renewcommand{\cftdot}{.}
\renewcommand{\cftsecleader}{\cftdotfill{\cftdotsep}}

% HEADER & FOOTER

\setlength{\headheight}{15pt}
\pagestyle{fancy}
\fancyhf{}
\renewcommand{\headrulewidth}{0.25pt}
\lhead{J. Scerri}
\chead{ICT1018 --- Coursework}
\rhead{\thepage}
% \cfoot{\pgfornament[scale=.35]{84}}

% TITLE

\title{ICT1018 --- Data Structures \& Algorithms 1\\
\vspace{1em}\textbf{Coursework}}

\date{\today}

\author {{\textbf{Juan Scerri}}\\
B.Sc. (Hons)(Melit.) Computing Science and Mathematics (First Year)}

\begin{document}

%----------------------------------
%	TITLE PAGE
%----------------------------------

\maketitle % Print the title page

\thispagestyle{empty} % Suppress headers and footers on the title page

%----------------------------------

\tableofcontents
\clearpage

% \lstlistoflistings\\

\section{Plagiarism Declaration}

Plagiarism is defined as \textit{``the unacknowledged use, as
one's own, of work of another person, whether or not such work
has been published, and as may be further elaborated in Faculty
or University guidelines''} (\ul{University Assessment
Regulations}, 2009, Regulation 39 (b)(i), University of Malta).

I, the undersigned, declare that the report submitted is my
work, except where acknowledged and referenced. I understand
that the penalties for committing a breach of the regulations
include loss of marks; cancellation of examination results;
enforced suspension of studies; or expulsion from the degree
programme.

Work submitted without this signed declaration will not be
corrected, and will be given zero marks.

\vfill

\begin{minipage}[t]{0.3\textwidth}
\ul{Juan Scerri} \medskip

\textbf{Student's full name} \medskip
\end{minipage}
\hfill
\begin{minipage}[t]{0.3\textwidth}
\ul{ICT1018} \medskip

\textbf{Study-unit code} \medskip
\end{minipage}
\hfill
\begin{minipage}[t]{0.3\textwidth}
\ul{{\today}} \medskip

\textbf{Date of submission} \medskip
\end{minipage}

\vspace{2cm}

\textbf{Title of submitted work:} \ul{Data Structures \&
Algorithms 1 Coursework}

\vspace{2cm}

\textbf{Student's signature} \medskip

\underline{\includegraphics[height=2cm]{sig}} \medskip

\section{Statement of Completion}

The statement of completion is a list of all the questions where
attempted, which work and which have bugs.

I, the undersigned, declare that the all questions have been
attempted. I also declare that all the solutions have been
tested and confirmed to work well (see list below).

\begin{itemize}
  \item \textit{Question 1} -- Attempted and works well.
  \item \textit{Question 2} -- Attempted and works well.
  \item \textit{Question 3} -- Attempted and works well.
  \item \textit{Question 4} -- Attempted and works well.
  \item \textit{Question 5} -- Attempted and works well.
  \item \textit{Question 6} -- Attempted and works well.
  \item \textit{Question 7} -- Attempted and works well.
  \item \textit{Question 8} -- Attempted and works well.
  \item \textit{Question 9} -- Attempted and works well.
  \item \textit{Question 10} -- Attempted and works well.
  \item \textit{Question 11} -- Attempted and works well.
  \item \textit{Question 12} -- Attempted and works well.
\end{itemize}

\vfill

\textbf{Student's Signature} \medskip

\underline{\includegraphics[height=2cm]{sig}} \medskip

Juan Scerri

\section{\textit{Question 1:} Shellsort \& Quicksort}

\begin{lstlisting}[language=Python]
def swap(a, x, y):
    t = a[x]
    a[x] = a[y]
    a[y] = t


# General term of the gap sequence.
# Proposed by Frank & Lazarus, 1960.
def next_h(n, m):
    return 2 * int(math.floor(n / 2 ** (m + 1))) + 1


def shellsort(a):
    n = len(a)
    m = 1
    h = next_h(n, m)

    while h >= 1:
        for i in range(h):
            # Modified insertion sort to
            # deal with gaps of size h.
            for j in range(i + h, n, h):
                for k in range(j, i, -h):
                    if a[k] < a[k - h]:
                        swap(a, k, k - h)
                    else:
                        break

        m += 1

        if h == 1:
            break

        h = next_h(n, m)

    return a
\end{lstlisting}

\section{\textit{Question 2:} Merging Sorted Arrays}
\section{\textit{Question 3:} Finding Extremes}
\section{\textit{Question 4:} Product Relation}
\section{\textit{Question 5:} \texttt{RPN} Calculator}
\section{\textit{Question 6:} Finding Primes}
\section{\textit{Question 7:} Building a Binary Search Tree}
\section{\textit{Question 8:} Square Root Approximation}
\section{\textit{Question 9:} Finding Duplicates}
\section{\textit{Question 10:} Recursive \texttt{max()}}
\section{\textit{Question 11:} Cosine/Sine Approximation}
\section{\textit{Question 12:} Fibonacci Sums}

\end{document}
