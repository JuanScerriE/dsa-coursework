\documentclass[12pt]{article}

% PACKAGES

% \usepackage[siunitx]{circuitikz}
% \usepackage{pgfornament}
% \usepackage{fancyvrb}
\usepackage[
top=2.50cm,
bottom=2.50cm,
left=2cm,
right=2cm,
marginparsep=0pt,
marginparwidth=0pt]{geometry}
\usepackage{fancyhdr}
\usepackage{float}
\usepackage{cancel}
\usepackage{mathtools}
\usepackage{amsmath}
\usepackage{amsthm}
\usepackage{amssymb}
\usepackage{textcomp}
\usepackage{ulem}
\usepackage{contour}
\usepackage{graphicx}
\usepackage{svg}
\usepackage{xcolor}
\usepackage[T1]{fontenc}
\usepackage[utf8]{inputenx}
\usepackage[unicode]{hyperref}
\usepackage[shortlabels]{enumitem}
\usepackage{booktabs}
\usepackage{bookmark}
\usepackage{listings}
\usepackage{xcolor}
\usepackage{tocloft}
\usepackage[backend=biber]{biblatex}

% MACROS & DEFS

\DeclareRobustCommand{\ul}[1]{%
	\uline{\phantom{#1}}%
	\llap{\contour{white}{#1}}%
}

\renewcommand{\ULdepth}{1.8pt}
\contourlength{0.8pt}

\setlength{\parindent}{0em}
\setlength{\parskip}{0.75em}

\definecolor{codegreen}{RGB}{0,135,0}
\definecolor{codegray}{RGB}{135,135,135}
\definecolor{codemagenta}{RGB}{215,0,135}
\definecolor{codepurple}{RGB}{135,0,175}
\definecolor{backcolour}{RGB}{238,238,238}

% PACKAGE CONFIG

% \usetikzlibrary{calc, arrows, positioning, circuits.logic.US}
\graphicspath{ {./images/} }

% Biblography
\addbibresource{dsa.bib}

\lstdefinestyle{code}{
	basicstyle=\ttfamily\small,
	commentstyle=\color{codegray}\itshape,
	keywordstyle=\color{codepurple},
	stringstyle=\color{codegreen},
	aboveskip=15pt,
	captionpos=b,
	abovecaptionskip=12.5pt,
	breaklines=true,
	numbers=none,
	frame=tb,
	framesep=5pt,
	keepspaces=true,
	showspaces=false,
	showstringspaces=false,
	breakatwhitespace=false,
	tabsize=2,
	showtabs=false,
}

\lstset{style=code}

% Set dots for table of contents
\renewcommand{\cftdot}{.}
\renewcommand{\cftsecleader}{\cftdotfill{\cftdotsep}}

% Set theorem
\newtheorem*{definition}{Definition}

% HEADER & FOOTER

\setlength{\headheight}{15pt}
\pagestyle{fancy}
\renewcommand{\headrulewidth}{0pt}
\lhead{J. Scerri}
\chead{ICT1018 --- Coursework}
\rhead{\thepage}

% TITLE

\title{ICT1018 --- Data Structures \& Algorithms 1\\
\vspace{1em}\textbf{Coursework}}

\date{\today}

\author {{\textbf{Juan Scerri}}\\
B.Sc. (Hons)(Melit.) Computing Science and Mathematics (First Year)}

\begin{document}

%----------------------------------
%	TITLE PAGE
%----------------------------------

\maketitle % Print the title page

\thispagestyle{empty} % Suppress headers and footers on the title page

%----------------------------------

\tableofcontents

\clearpage

\lstlistoflistings

\clearpage

\section{Plagiarism Declaration}

Plagiarism is defined as \textit{``the unacknowledged use, as
one's own, of work of another person, whether or not such work
has been published, and as may be further elaborated in Faculty
or University guidelines''} (\ul{University Assessment
Regulations}, 2009, Regulation 39 (b)(i), University of Malta).

I, the undersigned, declare that the report submitted is my
work, except where acknowledged and referenced. I understand
that the penalties for committing a breach of the regulations
include loss of marks; cancellation of examination results;
enforced suspension of studies; or expulsion from the degree
programme.

Work submitted without this signed declaration will not be
corrected, and will be given zero marks.

\vfill

\begin{minipage}[t]{0.3\textwidth}
\ul{Juan Scerri} \medskip

\textbf{Student's full name} \medskip
\end{minipage}
\hfill
\begin{minipage}[t]{0.3\textwidth}
\ul{ICT1018} \medskip

\textbf{Study-unit code} \medskip
\end{minipage}
\hfill
\begin{minipage}[t]{0.3\textwidth}
\ul{{\today}} \medskip

\textbf{Date of submission} \medskip
\end{minipage}

\vspace{2cm}

\textbf{Title of submitted work:} \ul{Data Structures \&
Algorithms 1 Coursework}

\vspace{2cm}

\textbf{Student's signature} \medskip

\underline{\includegraphics[height=2cm]{sig}} \medskip

\section{Statement of Completion}

The statement of completion is a list of all the questions where
attempted, which work and which have bugs.

I, the undersigned, declare that the all questions have been
attempted. I also declare that all the solutions have been
tested and confirmed to work well (see list below).

\begin{itemize}
  \item \textit{Question 1} -- Attempted and works well.
  \item \textit{Question 2} -- Attempted and works well.
  \item \textit{Question 3} -- Attempted and works well.
  \item \textit{Question 4} -- Attempted and works well.
  \item \textit{Question 5} -- Attempted and works well.
  \item \textit{Question 6} -- Attempted and works well.
  \item \textit{Question 7} -- Attempted and works well.
  \item \textit{Question 8} -- Attempted and works well.
  \item \textit{Question 9} -- Attempted and works well.
  \item \textit{Question 10} -- Attempted and works well.
  \item \textit{Question 11} -- Attempted and works well.
  \item \textit{Question 12} -- Attempted and works well.
\end{itemize}

\vfill

\textbf{Student's Signature} \medskip

\underline{\includegraphics[height=2cm]{sig}} \medskip

Juan Scerri

\section{Language \& Testing Methodology}

The programming language which I decided to choose to do this
assignment is Python 3. The main reason for chosing Python 3 was
the speed of development.

Each algorithm was tested using a separarte file, in general
named \texttt{test\_q(n).py}, where \texttt{n} refers to the
question number. Each file contains the expected results of the
algorithms. Each file compares the expected results with the
results of the algorithm. If the results match ``\texttt{Test
Passed}'' will be printed otherwise ``\texttt{Test Failed}''
will be printed. 

Moreover, some usage of the algorithms was implemented in the
test files. For example \textit{Question 1} required that we
sort two arrays which have at least 256 elements and have a
different length. Such functionality was implemented in the test
files and can be seen in the dump screen for each question.

\section{\textit{Question 1:} Shellsort \& Quicksort}

The gap sequence used in the Shellsort algorithm was proposed by
\textcite{frank60}, and the partition scheme used in the
Quicksort algorithm was proposed by \textcite{hoare62}.

\lstinputlisting[caption={Answer to \textit{Question
1}},language=Python]{answers/q1.py}

\lstinputlisting[caption={Test for \textit{Question 1}
},language=Python]{test_q1.py}

\lstinputlisting[caption={Screen dump for \textit{Question
1}}]{dumps/dump_q1.txt}

\section{\textit{Question 2:} Merging Sorted Arrays}

\lstinputlisting[caption={Answer to \textit{Question
2}},language=Python]{answers/q2.py}

\lstinputlisting[caption={Test for \textit{Question 2}
},language=Python]{test_q2.py}

\lstinputlisting[caption={Screen dump for \textit{Question
2}}]{dumps/dump_q2.txt}

\section{\textit{Question 3:} Finding Extremes}

To show that if an array does not have extremes then it must be
stored we need to define what it means for an array to be sorted
and what it means for an array to have extremes.

\begin{definition}
  A zero indexed array of size $n$, $A$, is said to be
  \textbf{sorted} if
  \begin{align*}
    \forall \, i \in \{1,2,3,\ldots,n-1\} :
    A[i-1] \leq A[i] \leq A[i+1] \\
    \vee \,
    A[i-1] \geq A[i] \geq A[i+1]
  \end{align*}
\end{definition}

\begin{definition}
  A zero indexed array of size $n$, $A$, is said to have
  \textbf{extremes} if
  \begin{align*}
    \exists \, i \in \{1,2,3,\ldots,n-1\} :
    (A[i-1] > A[i] \, \wedge \, A[i+1] > A[i]) \\ 
    \vee \,
    (A[i-1] < A[i] \, \wedge \, A[i+1] < A[i])
  \end{align*}
\end{definition}

Now we simply show that these two statements are the negation of
each other. We arbitrarily decide to negate the first (i.e. the
definition of \textbf{sorted}).

\begin{align*}
  \neg \, \forall \, i \in \{1,2,3,\ldots,n-1\} :
  A[i-1] \leq A[i] \leq A[i+1] \\
  \vee \,
  A[i-1] \geq A[i] \geq A[i+1] \\\\
  \Longleftrightarrow \, \exists \, i \in \{1,2,3,\ldots,n-1\} :
  (A[i-1] > A[i] \, \vee \, A[i] > A[i+1]) \\
  \wedge \,
  (A[i-1] < A[i] \, \vee \, A[i] < A[i+1]) \\\\
  \Longleftrightarrow \, \exists \, i \in \{1,2,3,\ldots,n-1\} :
  (A[i-1] > A[i] \, \wedge \, A[i-1] < A[i]) \\
  \vee \,
  (A[i-1] > A[i] \, \wedge \, A[i] < A[i+1]) \\
  \vee \,
  (A[i] > A[i + 1] \, \wedge \, A[i-1] < A[i]) \\
  \vee \,
  (A[i] > A[i + 1] \, \wedge \, A[i] < A[i+1]) \\\\
  \Longleftrightarrow \, \exists \, i \in \{1,2,3,\ldots,n-1\} :
  \text{False} \\
  \vee \,
  (A[i-1] > A[i] \, \wedge \, A[i] < A[i+1]) \\
  \vee \,
  (A[i-1] < A[i] \, \wedge \, A[i] > A[i + 1]) \\
  \vee \,
  \text{False} \\\\
  \Longleftrightarrow \, \exists \, i \in \{1,2,3,\ldots,n-1\} :
  (A[i-1] > A[i] \, \wedge \, A[i] < A[i+1]) \\
  \vee \,
  (A[i-1] < A[i] \, \wedge \, A[i] > A[i + 1])
\end{align*}

Therefore, an array has \textbf{extremes} if and only if it is
\ul{not} \textbf{sorted}. Hence, if an array does \ul{not} have
\textbf{extremes} then it must be \textbf{sorted}.

\lstinputlisting[caption={Answer to \textit{Question
3}},language=Python]{answers/q3.py}

\lstinputlisting[caption={Test for \textit{Question 3}
},language=Python]{test_q3.py}

\lstinputlisting[caption={Screen dump for \textit{Question
3}}]{dumps/dump_q3.txt}

\section{\textit{Question 4:} Finding Product Relations}

\lstinputlisting[caption={Answer to \textit{Question
4}},language=Python]{answers/q4.py}

\lstinputlisting[caption={Test for \textit{Question 4}
},language=Python]{test_q4.py}

\lstinputlisting[caption={Screen dump for \textit{Question
4}}]{dumps/dump_q4.txt}

\section{\textit{Question 5:} \texttt{RPN} Calculator}

\lstinputlisting[caption={Answer to \textit{Question
5}},language=Python]{answers/q5.py}

\lstinputlisting[caption={Test for \textit{Question 5}
},language=Python]{test_q5.py}

\lstinputlisting[caption={Screen dump for \textit{Question
5}}]{dumps/dump_q5.txt}

\section{\textit{Question 6:} Finding Primes}

The algorithm used for the sieve was inspired by
\textcite{sorenson90}. For the optimisation read the comments in
the code.

\lstinputlisting[caption={Answer to \textit{Question
6}},language=Python]{answers/q6.py}

\lstinputlisting[caption={Test for \textit{Question 6}
},language=Python]{test_q6.py}

\lstinputlisting[caption={Screen dump for \textit{Question
6}}]{dumps/dump_q6.txt}

\section{\textit{Question 7:} Building a Binary Search Tree}

\lstinputlisting[caption={Answer to \textit{Question
7}},language=Python]{answers/q7.py}

\lstinputlisting[caption={Test for \textit{Question 7}
},language=Python]{test_q7.py}

\lstinputlisting[caption={Screen dump for \textit{Question
7}}]{dumps/dump_q7.txt}

\section{\textit{Question 8:} Square Root Approximation}

\lstinputlisting[caption={Answer to \textit{Question
8}},language=Python]{answers/q8.py}

\lstinputlisting[caption={Test for \textit{Question 8}
},language=Python]{test_q8.py}

\lstinputlisting[caption={Screen dump for \textit{Question
8}}]{dumps/dump_q8.txt}

\section{\textit{Question 9:} Finding Duplicates}

\lstinputlisting[caption={Answer to \textit{Question
9}},language=Python]{answers/q9.py}

\lstinputlisting[caption={Test for \textit{Question 9}
},language=Python]{test_q9.py}

\lstinputlisting[caption={Screen dump for \textit{Question
9}}]{dumps/dump_q9.txt}

\section{\textit{Question 10:} Recursive \texttt{max()}}

\lstinputlisting[caption={Answer to \textit{Question
10}},language=Python]{answers/q10.py}

\lstinputlisting[caption={Test for \textit{Question 10}
},language=Python]{test_q10.py}

\lstinputlisting[caption={Screen dump for \textit{Question
10}}]{dumps/dump_q10.txt}

\section{\textit{Question 11:} Cosine/Sine Approximation}

\lstinputlisting[caption={Answer to \textit{Question
11}},language=Python]{answers/q11.py}

\lstinputlisting[caption={Test for \textit{Question 11}
},language=Python]{test_q11.py}

\lstinputlisting[caption={Screen dump for \textit{Question
11}}]{dumps/dump_q11.txt}

\section{\textit{Question 12:} Fibonacci Sums}

\lstinputlisting[caption={Answer to \textit{Question
12}},language=Python]{answers/q12.py}

\lstinputlisting[caption={Test for \textit{Question 12}
},language=Python]{test_q12.py}

\lstinputlisting[caption={Screen dump for \textit{Question
12}}]{dumps/dump_q12.txt}

\printbibliography

\end{document}
